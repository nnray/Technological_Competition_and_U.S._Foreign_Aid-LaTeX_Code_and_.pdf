\section*{Theory} 
U.S. wants to protect American Big Tech markets that are potentially going to be lost in a zero-sum game with Chinese companies (if China successfully shares tech and builds infrastructure).

\begin{align*}
    H_1a:\; & \text{A pattern should be evident of U.S. money flowing into places where either}\\
    & \text{Big Tech markets are OR}\\
    H_1b:\; & \text{Where Chinese influence/money is increasing OR}\\
    H_2:\; & \text{Where Chinese influence is NOT increasing (U.S. defending markets)}
\end{align*}

\section*{Alternative Explanations}
The U.S. is not concerned necessarily about Big Tech markets, but instead increasing aid in an attempt to thwart China's goals of economic growth abroad.

\textbf{Implication}
\begin{align*}
    H_3:\; & \text{There should be a relatively random pattern of spending or at least}\\
    & \text{a pattern not associated with Big Tech markets}
\end{align*}

The U.S. is not necessarily concerned about market access for American Big Tech or China's goal of increasing economic growth through trade/economic diplomacy, but rather trying to counter Chinese money's influence on digital authoritarianism. The U.S. would then be trying to share it's own tech and build it's own internet infrastructure to have more control over the recipient's internet freedom levels.

\textbf{Implication}
\begin{align*}
    H_4:\; & \text{U.S. spending patters would be associated with increasing}\\
    & \text{internet freedom levels and potentially concentrated in places that are}\\
    & \text{not already receiving tech and infrastructure from China}\\
\end{align*}