Scholars have long debated over the determinants of foreign aid and its effectiveness, generally concluding that foreign aid is a strategic tool for donors (e.g., \cite{alesina2000}; \cite{hoeffler2011}) and that its effectiveness greatly varies (e.g., see \cite{christensen2011}). Most recently, there has been a resurgence in the exploration of aid effectiveness due to an unprecedented availability of Chinese foreign aid. With regard to the U.S., for example, researchers have found that U.S. aid has become more effective at stimulating liberal democratic values (\cite{blair2022}), especially in countries not receiving large amounts of Chinese aid (\cite{dreher2021}).

However, there has been a lack of equal reconsideration for how Chinese aid potentially influences the strategic aims of donors. I argue that the abundance of Chinese money should alter donor strategy, especially for the U.S. Specifically, if the U.S. is a rational actor, it should place a greater emphasis on a recipient's ``internet freedom'' level when allocating aid in light of China's rising spending influence. This is rational because 1) advancing global internet freedom in opposition to China's model of the internet has been a vocal aspect of U.S. foreign policy for decades (e.g., \cite{government2010}, \cite{government2022b}) and 2) promoting internet freedom could be an effective use of resources given the findings on U.S. aid's ability to encourage liberal values in the contemporary aid environment (\cite{blair2022}).

This logic leads to the following hypothesis:
\begin{align*}
    H1 &:\text{U.S. aid should be related to a recipient's internet freedom level and the amount of Chinese}\\
    &\text{aid they receive.}
\end{align*}

I plan to use OLS with country-year fixed effects to evaluate how much of the variation in U.S. foreign aid to Africa from 2010 to 2017 can be explained by an interaction between Chinese foreign aid and recipient internet freedom levels. Internet freedom is measured by the Digital Society Project (DSP), using V-Dem methodology and expert-coded surveys to analyze how social media is used as a political tool by countries around the world (\cite{mechkova2022}). Important controls include a country's GDP per capita and an indicator for internet development.