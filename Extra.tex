%----------------------------------------------------------
%To do
Case study comparing U.S. aid changes to countries that had similar internet freedom levels but experienced different influxes of Chinese aid? Need to do case studies of results and examples, in general. 
%----------------------------------------------------------
%Rough Draft extras
\section*{Literature Review}
Foreign aid before BRI (a few sentences with biggest takeaways; self-interest, ineffectiveness)

BRI (describe, including sharing of technology or suspicion of digital authoritarianism)

Foreign aid literature after BRI (a few paragraphs, highlighting how effectiveness has increased and evidence that donors are adjusting, i.e. lower conditionality for example)

\section*{Theory}
To explain how other donor countries should be expected to react to China's recent dominance of the international financing arena, I model a simple entry game between China and an unspecified country. Once China decides to be heavily involved in foreign finance, 

Essentially, the payoff a traditional donor gets from increasing their foreign aid spending given that China has ``entered'' the arena is dependent on their level of competition with China. I focus on competition along the axis of internet influence...? Why this rather than other axes? Does it actually capture economic competition, too, as I would want to argue? (Foreign aid is self-interested and seemingly effective in this environment, so it seems reasonable that a policy emphasis on IF would trickle down to aid spending... right? ;Perhaps China spreads technology/domestic companies for economic reasons, which could decrease IF and U.S. would oppose to not lose market access (Facebook) and would try to bolster IF)...

Where $\gamma$ is the influence that $Donor\;1$ aid has on internet freedom or technology or the internet, and $\delta$ is the ability for $Donor\;2$ to counter this influence or to influence IF, technology, or the internet themselves. Then focus on the detailing these assumptions.

\section*{Conclusion}
Crazy conclusion: I suspect that the U.S. and China are in a technological proxy war. China, in an effort to boost its global influence and economic growth, is attempting to build digital infrastructure around the world and encouraging the use of Chinese technology. In response, the U.S. is bankrolling it's own digital infrastructure projects out of fear that its companies will be locked out of markets that subscribe to Chinese technology. The setting of this conflict is mainly developing countries that are open to receiving aid from these two countries. The U.S. prefers countries to have a liberal view of the internet, thereby facilitating the growth of Western technology companies. These developing countries, however, have an incentive to harness the economic benefits of the internet while exercising more control over the potential political costs- therefore, they might welcome the Chinese model. The Chinese, meanwhile, don't really care but are just aiming to expand their consumer base, but the Chinese model has illiberal externalities.

%----------------------------------------------------------
%Old theory ideas (explaining an expected relationship)
\section*{Theory} 
U.S. wants to protect American Big Tech markets that are potentially going to be lost in a zero-sum game with Chinese companies (if China successfully shares tech and builds infrastructure).

\begin{align*}
    H_1a:\; & \text{A pattern should be evident of U.S. money flowing into places where either}\\
    & \text{Big Tech markets are OR}\\
    H_1b:\; & \text{Where Chinese influence/money is increasing OR}\\
    H_2:\; & \text{Where Chinese influence is NOT increasing (U.S. defending markets)}
\end{align*}

\section*{Alternative Explanations}
1) The U.S. is not concerned necessarily about Big Tech markets, but instead increasing aid in an attempt to thwart China's goals of economic growth abroad.

\textbf{Implication}
\begin{align*}
    H_3:\; & \text{There should be a relatively random pattern of spending or at least}\\
    & \text{a pattern not associated with Big Tech markets}
\end{align*}

2) The U.S. is not necessarily concerned about market access for American Big Tech or China's goal of increasing economic growth through trade/economic diplomacy, but rather trying to counter Chinese money's influence on digital authoritarianism. The U.S. would then be trying to share it's own tech and build it's own internet infrastructure to have more control over the recipient's internet freedom levels.

\textbf{Implication}
\begin{align*}
    H_4:\; & \text{U.S. spending patters would be associated with increasing}\\
    & \text{internet freedom levels and potentially concentrated in places that are}\\
    & \text{not already receiving tech and infrastructure from China}\\
\end{align*}

9/28
If U.S. foreign aid policy is rational and related to internet freedom, then the U.S. must stand to gain from spending money on improving others' internet freedom. I argue that the U.S. benefits in two main ways from supporting internet freedom: protecting American technology's access to developing markets (possibly being encouraged by China to use Chinese technology instead) while ensuring that China faces resistance in continuing its economic growth (which is it's main goal of this project).

If this is true, then we should observe U.S. foreign aid spending patterns with one set of the following hypotheses:

\begin{align*}
    H_1a:\; & \text{U.S. aid should be associated with countries with high levels of internet freedom.}\\
    H_1b:\; & \text{U.S. aid should be associated with countries with low levels of internet freedom.}\\
    H_2a:\; & \text{U.S. aid should be associated with countries receiving high amounts of Chinese aid.}\\
    H_2b:\; & \text{U.S. aid should be associated with countries receiving low amounts of Chinese aid.}
\end{align*}

9/29
\section*{Theory}
Depending on how the literature review goes, we can mention our theory of why it is rational for the U.S. to engage in promoting internet freedom (aid effectually increases democracy, and democracy is good for business (American Big Tech)).

%----------------------------------------------------------