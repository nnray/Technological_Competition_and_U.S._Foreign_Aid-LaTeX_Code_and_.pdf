\section{Final Draft Extras}
\subsection{Introduction}
The People's Republic of China (China) has become the ``lender of first resort'' for developing countries in the last decade \parencite[1]{dreher2022}.\footnote{All data and code can be found on GitHub at https://github.com/nnray/Analysis. In the ``Code'' folder, navigate to ``628.Rmd'' for my R script. My full \LaTeX \;file can be found at https://github.com/nnray/628 under ``Paper.tex''.} An expanse of recent work has begun documenting how our understanding of global finance has changed since China's financial preeminence. Some of this research looks at how Western money, both foreign aid and loans, has been affected by the unprecedented availability of Chinese money (e.g., \cite{hernandez2017}). With regard to the United States (U.S.), there is evidence that its foreign aid has become more effective at stimulating liberal democratic values among recipient citizens (\cite{blair2022}). %other work

Surprisingly, though, there has been relatively little work on how U.S. spending behavior itself may have changed with the rise of Chinese finance. In this article I aim to address this shortcoming and explore how U.S. foreign aid spending has responded to a global increase in Chinese finance. Focusing on Africa, I expect that an empirical relationship between U.S. aid and internet governance has emerged as a result of Chinese spending. Using the most rigorous measure of internet governance available, I find evidence that internet governance describes the variation in U.S. foreign aid spending after accounting for common measures of regime type, economic development, and internet development.

I make two contributions in the uncovering of this relationship. Firstly, I add to the foreign aid literature by detailing a novel determinant of U.S. foreign aid and providing another clue on how Chinese money may have impacted Western donor behavior more generally. Secondly, I speak to literature on the liberal international order (LIO) and the concern that China may undermine tenets of the LIO. %expand on this a bit?

I proceed as follows. In the next section, I define internet governance and detail the theoretical reasoning behind its importance and relationship to U.S. foreign aid. After developing hypotheses, I describe my empirical strategy and data. My findings and a discussion of their implications are presented before concluding.

\subsection{Theory}
Internet governance was defined in the second phase of the World Summit on the Information Society (WSIS), a meeting led by the United Nations' (UN) International Telecommunication Union (ITU), which was held in Tunis, Tunisia in 2005. It is ``the development and application by governments, the private sector and civil society, in their respective roles, of shared principles, norms, rules, decision-making procedures, and programmes that shape the evolution and use of the Internet'' (World Summit on the Information Society 2005, item 34). Essentially, internet governance refers to the way in which governments, firms, and citizens interact with the internet. %What it composes of and commmon measures... 

But why focus on internet governance? What can be uniquely gained in looking at U.S. foreign aid through a potentially new determinant? The U.S. and China have recently been observed to direct a substantial amount of money towards building digital infrastructure in the developing world, which consists of the cables and other hardware that facilitate internet connectivity. For example, (\cite{thewhitehouse2022}) and (\cite{shen2022}). 

I contend that these observations are difficult to fully explain by previous accounts. Past work has identified that foreign aid is dominated by the self-interests of donors and is not allocated according to recipient need (\cite{hoeffler2011}; others). That is, foreign aid expenditures on digital infrastructure is likely in the foreign policy interests of the U.S.. However, it is not ex ante obvious how exactly this is the case. The U.S. has historically prioritized economic growth (\cite{dreher2021}; others), but a desire to stimulate economic growth does not explain the adversarial nature of this recent digital infrastructure aid and the apparently zero-sum stance against China. Indeed it can be argued that Chinese investment is in the U.S.'s interests if those interests are purely concerned with economic growth. Political competition likely is involved here, but it has yet to be shown how investing in digital infrastructure is related to the political competition between the U.S. and China, although it is possible that the U.S. has returned to a Cold-War style of foreign policy. Also, the U.S. has previously proven that they are perfectly willing to aid those with illiberal political models (Saudi Arabia?). Consequently, I speculate that an adequate explanation of the recent competition regarding digital infrastructure likely revolves around the digital aspect itself, and the U.S. China competition over technological superiority. This is manifested in concern for internet governance. Thus, I argue that U.S. foreign aid should be empirically related to internet governance for the following reasons.

Summarily, it is rational for the U.S. to condition its foreign aid spending on a recipient's internet governance in response to a rise in Chinese money because: 1) the U.S. has increasingly voiced a preference for more liberal internet governance worldwide; 2) U.S. foreign aid has been shown to be effective at promoting liberal values; 3) the Chinese are suspected of sharing technology with recipient countries and negatively influencing internet governance; and 4) the U.S. and China have been engaged in a competition for technological superiority, of which the internet is a part of. To the extent that these statements are true, together they imply that it makes sense for the U.S. to spend aid in a way that positively influences the liberalness of internet governance and inhibits the spread of Chinese technology and influence. I will detail the support behind these statements in the following paragraphs.

Firstly, there is qualitative evidence that the U.S. has a preference for liberal internet governance in other countries. In 1997 President Bill Clinton exhorted the world to adopt a laissez-faire, commercially friendly approach to the internet (The White House 1997, principles 1-5) and by the 2000's, these preferences crystallized into foreign policy initiatives as so-called ``internet freedom'' programs were initiated by the U.S. Department of State (e.g., \cite{henry2014}; \cite{u.s.departmentofstate2021}). U.S. attention to internet governance has continued to grow, with Secretary of State Hillary Clinton outlining U.S. demands for a ``free'' internet (\cite{u.s.departmentofstate2010}) and the U.S. Senate considering additional funding for internet freedom programs in 2022 (\cite{foreignrelationscommittee2022}). In the same year, a declaration led by the U.S. was signed by over 60 countries to combat global ``digital authoritarianism'' (\cite{u.s.departmentofstate2022}).

Secondly, there is indirect evidence that foreign aid is an effective tool for influencing internet politics in recipient countries. For instance, in Africa U.S. aid is found to be relatively successful in promoting liberal democratic values amongst Afrobarometer respondents when compared to Chinese foreign aid \cite{blair2022}. %bearce and tirone, dreher et al, etc

Thirdly, there is reason to believe that Chinese aid could influence a recipient's political use of the internet in an illiberal manner. China launched the Belt and Road Initiative (BRI) in 2013 and the Digital Silk Road (DSR) in 2015, both aimed at enlarging China's global influence while strengthening economic growth (\cite{dreher2022}). In doing so, it is alleged that China is sharing technology with recipient states that could improve state capacity and allow recipients to curb free expression, both digitally and physically (\cite{hillman2021}). Other work, though, rightly points out that China is a non-unitary actor with private firms that could undermine a state-led technology sharing campaign (\cite{shen2018a}) and that China's technology spending has not changed much immediately following the announcement of the DSR (\cite{tugendhat2021}). The West believes China to be spreading its illiberal model of the internet (e.g., \cite{hillman2021}; \cite{u.s.departmentofstate2022}), or at least that there is a good chance for it to (\cite{triolo2020}).

Fourthly, it has been observed by some experts that the U.S. and China are competing for global technological superiority (\cite{hass2021}).

Convincingly explain how these four statements lead one to expect that U.S. aid is empirically related to internet governance. Describe diagram 1. Then state this expectation formally, think about directionality of hypotheses and how to base them more off the literature (effectiveness, etc.).

\pagebreak
%may add Time 0 about Soviet influence to emphasize the similarity here in U.S. response to China?
\subsection*{Diagram 1}
    %Time 0
    \begingroup
    \leftskip=0cm plus 0.5fil \rightskip=0cm plus -0.5fil
    \parfillskip=0cm plus 1fil
    Time 1 (1990's - 2000's)\par
    \endgroup
    \begin{figure}[h]
    \centering
    \begin{tikzpicture}
    [node distance = 1cm, auto,font=\footnotesize,
    % STYLES
    every node/.style={node distance=3.5cm},
    % The comment style is used to describe the characteristics of each force
    comment/.style={rectangle, inner sep= 5pt, text width=4cm, node distance=0.25cm, font=\scriptsize\sffamily},
    % The force style is used to draw the forces' name
    force/.style={rectangle, draw, fill=black!10, inner sep=5pt, text width=4cm, text badly centered, minimum height=1.2cm, font=\bfseries\footnotesize\sffamily}]
    % Draw forces
    \node [force] (usaid) {U.S. Aid};
    \node [force, above of=usaid] (effectiveness) {Effectiveness};
    \node [force, left=1cm of effectiveness] (preferences) {Preference for Politico-economic Liberalism};
    %preferences
    \node [comment, below=0.25cm of preferences] {Preference for a liberal internet largely subsumed under politico-economic liberalism and not independently influential};
    %effectiveness
    \node [comment, below=0.2cm of effectiveness] {Aid positively effective on economic growth (Bearce and Tirone 2010)};
    %usaid
    \node [comment, below=0.25cm of usaid] {Aid primarily directed to recipients with more liberal politico-economic structures};
    % Draw the links between forces
    \path[->,thick] 
    (effectiveness) edge (usaid)
    (preferences) edge (usaid);
    \end{tikzpicture} 
    \end{figure}

    %Time 1
    \begingroup
    \leftskip=0cm plus 0.5fil \rightskip=0cm plus -0.5fil
    \parfillskip=0cm plus 1fil
    Time 2 (2010's)\par
    \endgroup
    \begin{figure}[h]
    \centering
    \begin{tikzpicture}
    [node distance = 1cm, auto,font=\footnotesize,
    % STYLES
    every node/.style={node distance=3.5cm},
    % The comment style is used to describe the characteristics of each force
    comment/.style={rectangle, inner sep= 5pt, text width=4cm, node distance=0.25cm, font=\scriptsize\sffamily},
    % The force style is used to draw the forces' name
    force/.style={rectangle, draw, fill=black!10, inner sep=5pt, text width=4cm, text badly centered, minimum height=1.2cm, font=\bfseries\footnotesize\sffamily}]
    % Draw forces
    \node [force] (usaid) {U.S. Aid};
    \node [force, above of=usaid] (effectiveness) {Effectiveness};
    \node [force, left=1cm of effectiveness] (preferences) {Preference for Politico-economic Liberalism \underline{and} a Liberal Internet};
    \node [force, above of=preferences] (caid) {Chinese Aid};
    %preferences
    \node [comment, below=0.25cm of preferences] {Preference for a liberal internet largely now independently influential due to technology competition with China};
    %effectiveness
    \node [comment, below=0.25cm of effectiveness] {Aid now positively effective at promoting liberal values in light of Chinese aid};
    %usaid
    \node [comment, below=0.25cm of usaid] {Aid now related to a recipient's internet liberalness};
    %caid
    \node [comment, below=0.25cm of caid] {Chinese launched the BRI in 2013 but had been increasing its foreign aid before then};
    % Draw the links between forces
    \path[->,thick] 
    (effectiveness) edge (usaid)
    (preferences) edge (usaid)
    (caid) edge (preferences)
    (caid) edge (effectiveness);
    \end{tikzpicture}
    \end{figure}

    %Time 2
\pagebreak

\section{Rough Draft Extras}
\subsection*{Literature Review}
Foreign aid before BRI (a few sentences with biggest takeaways; self-interest, ineffectiveness)

BRI (describe, including sharing of technology or suspicion of digital authoritarianism)

Foreign aid literature after BRI (a few paragraphs, highlighting how effectiveness has increased and evidence that donors are adjusting, i.e. lower conditionality for example)

\subsection*{Theory}
To explain how other donor countries should be expected to react to China's recent dominance of the international financing arena, I model a simple entry game between China and an unspecified country. Once China decides to be heavily involved in foreign finance, 

Essentially, the payoff a traditional donor gets from increasing their foreign aid spending given that China has ``entered'' the arena is dependent on their level of competition with China. I focus on competition along the axis of internet influence...? Why this rather than other axes? Does it actually capture economic competition, too, as I would want to argue? (Foreign aid is self-interested and seemingly effective in this environment, so it seems reasonable that a policy emphasis on IF would trickle down to aid spending... right? ;Perhaps China spreads technology/domestic companies for economic reasons, which could decrease IF and U.S. would oppose to not lose market access (Facebook) and would try to bolster IF)...

Where $\gamma$ is the influence that $Donor\;1$ aid has on internet freedom or technology or the internet, and $\delta$ is the ability for $Donor\;2$ to counter this influence or to influence IF, technology, or the internet themselves. Then focus on the detailing these assumptions.

\subsection*{Model}
I argue that it is rational for the U.S. to consider the internet when distributing foreign aid and, specifically, that a recipient's current use of the internet should be a strong predictor for receiving U.S. funds. To explore the rationality of this alleged behavior, I consider a simple entry deterrence game. China is the first-mover and decides to enter or spend foreign aid in a particular country by allocating aid or not. The U.S. then moves second, making a similar decision to distribute foreign aid depending on if China has entered or not. There are two parameters that decide the payoffs of the game: $\gamma$, the influence that Chinese aid has on a recipient country's internet model and $\delta$, the influence of U.S. aid. Low values represent high degrees of influence for both. Essentially, the strategies of the players are symmetrical, with each country deciding to spend aid if their degree of influence is relatively larger than their rival's.

When U.S. foreign aid is net effective at encouraging a liberal model of the internet in a country, it decides to allocate aid. China, observing the U.S.'s expenditure, decides not to distribute foreign aid in that country. However, if the U.S. concludes not to spend aid monies due to its aid being relatively ineffective at influencing a recipient's use of the internet, then China allocates aid and exerts its influence on a recipient's internet usage. The extensive-form of this game is displayed in Figure 1, with more detail given in the appendix.

\subsubsection*{Figure 1}
\begin{istgame}[scale=4]
\xtdistance{5mm}{15mm}
\istroot(0)(0,0){\textit{China}}
\istb{\neg Spend}[al]
\istb{Spend}[ar]{0\leq\gamma,c,\delta\leq1}[below,yshift=25mm,xshift=25mm]
\endist
\xtdistance{5mm}{10mm}
\istroot(1)(0-1)<above left>{\textit{U.S.}}
\istb{\neg Spend}[l]{(\gamma,\;\delta)}
\istb{Spend}[r]{(\gamma,\;\frac{1}{\delta}-c_2)}
\endist
\istroot(2)(0-2)<above right>{\textit{U.S.}}
\istb{\neg Spend}[l]{(\frac{1}{\gamma}-c_1,\;\delta)}
\istb{Spend}[r]{(\frac{\delta}{\gamma}-c_1,\;\frac{\gamma}{\delta}-c_2)}
\endist
\xtdistance{10mm}{10mm}
\end{istgame}

\subsection*{Notable Assumptions}
Ample explanation is in order. Firstly, there is a host of technical restrictions inherent to my argument. Other determinants of foreign aid, for example, are not considered here. Thus, any changes in spending should be interpreted as the U.S. or China deciding to spend on internet influence or not $ceteris\;paribus$, or holding other foreign aid avenues constant. 

These decisions are also stipulated as dichotomous, where China or the U.S. simply agree to spend foreign aid or not. In reality, foreign aid spending is clearly continuous in nature and, moreover, typically based on repeated interactions between both other donors and recipients. This is also a full information model, with no uncertainty present over players' decisions to spend. I further assume that payoffs from the decision to spend or not are zero-sum at the extremes, where an arbitrarily large influence by one country dwarfs the incentives for the other to spend foreign aid.

Secondly, and most critically, there are a number of substantive assumptions invoked in this theoretical exposition that must be detailed. For one, I am assuming that the U.S. has a preference over internet governance around the world and, additionally, that foreign aid is an effective instrument for the U.S. to actualize this preference. Somewhat symmetrically, I assume that Chinese aid has an effect on a recipient's internet model and that this effect is negative or illiberal. I will take these assumptions in turn, connecting these behavioral claims to the established literature where possible.

\subsection*{Implications}
Returning to my theory, the model simplistically states that the U.S. and China should allocate foreign aid when aid is relatively effective at influencing a recipient's political model of the internet. Some of the literature implies that U.S. aid is effective when Chinese aid is low or not present (\cite{dreher2021}). Since it is assumed that Chinese aid has a negative effect on how recipients use the internet, this further implies that U.S. aid should be spent in countries with more liberal models of the internet to ensure maximum effectiveness. These concurrent implications lead to the following hypotheses:

\begin{align*}
    &\text{``Interstate Competition Hypothesis''}\\
    H_{1a}&:\;\text{U.S. aid inversely related to Chinese aid}\\
    &\text{``Trickle Down Hypothesis''}\\
    H_{2a}&:\;\text{U.S. aid directly related to recipient internet model}\\
\end{align*}

I label $H_{1a}$ the ``Interstate Competition Hypothesis'' because if U.S. foreign aid is truly most effective where Chinese aid is not present, this implies that the U.S. would respond to an increase in Chinese aid by focusing its spending in countries not currently receiving aid from China. China and the U.S. would thus be spending in completely separate countries or regions.

$H_{2a}$ is titled the ``Trickle Down Hypothesis'' since the U.S. would spend more foreign aid in countries with more liberal internet models, implying that the U.S.'s preference for a globally liberal internet model is manifested by it focusing on bolstering countries with already relatively liberal models.

However, it is also hypothetically possible that the U.S. would try to compete with Chinese aid by directly countering it in countries where the Chinese are active and generating intrastate competition. Similarly, it could make sense for the U.S. to be focusing on improving the internet in countries with relatively illiberal models, as it is plausible that foreign aid's influence on the internet would have higher marginal benefits. This opposing set of logic generates the following hypotheses:
\begin{align*}
    &\text{``Intrastate Competition Hypothesis''}\\
    H_{1b}&:\;\text{U.S. aid directly related to Chinese aid}\\
    &\text{``Bottom Up Hypothesis''}\\
    H_{2b}&:\;\text{U.S. aid inversely related to recipient internet model}\\
\end{align*}

\subsection*{Discussion}
test if Chinese aid related to IF itself, implications

\subsection*{Conclusion}
Crazy conclusion: I suspect that the U.S. and China are in a technological proxy war. China, in an effort to boost its global influence and economic growth, is attempting to build digital infrastructure around the world and encouraging the use of Chinese technology. In response, the U.S. is bankrolling it's own digital infrastructure projects out of fear that its companies will be locked out of markets that subscribe to Chinese technology. The setting of this conflict is mainly developing countries that are open to receiving aid from these two countries. The U.S. prefers countries to have a liberal view of the internet, thereby facilitating the growth of Western technology companies. These developing countries, however, have an incentive to harness the economic benefits of the internet while exercising more control over the potential political costs- therefore, they might welcome the Chinese model. The Chinese, meanwhile, don't really care but are just aiming to expand their consumer base, but the Chinese model has illiberal externalities.

\subsection*{Appendix}
The entry game described in the theory section (Figure 1) is solved via backwards induction. The parameter $\delta$ inversely represents the influence that U.S. aid has on a recipient's internet model while $\gamma$ inversely represents the influence of Chinese aid. In other words, low values of $\delta$ or $\gamma$ imply high degrees of influence. 

Hypothetically, if $China$ chooses to $Spend$, the $U.S.$ chooses to $Spend$ if and only if (iff):

\begin{tabular}{c l}
    $\frac{\gamma}{\delta}-c_2> \delta$ &  As $\delta$ increases, this condition becomes harder to satisfy.\\
\end{tabular}

Hypothetically, if $China$ chooses to $\neg Spend$, the $U.S.$ chooses to $Spend$ iff:

\begin{tabular}{c l}
    $\frac{1}{\delta}-c_2> \delta$ &  As $\delta$ increases, this condition becomes harder to satisfy.\\
\end{tabular}

Hypothetically, when $\delta$ is arbitrarily small, the $U.S.$ chooses to $Spend$ and $China$ $Spend$s iff:

\begin{tabular}{c l}
    $\frac{\delta}{\gamma}-c_1> \gamma$ &  As $\gamma$ increases, this condition becomes harder to satisfy.\\
\end{tabular}

Hypothetically, when $\delta$ is arbitrarily large, the $U.S.$ chooses to $\neg Spend$ and $China$ $Spend$s iff:

\begin{tabular}{c l}
    $\frac{1}{\gamma}-c_1> \gamma$ &  As $\gamma$ increases, this condition becomes harder to satisfy.\\
\end{tabular}

Summarily, if $\delta$ and $\gamma$ are arbitrarily small, both players will $Spend$. When $\delta$ is arbitrarily small and $\gamma$ is arbitrarily large, the $U.S.$ $Spend$s and $China$ $\neg Spend$. If $\delta$ is arbitrarily large and $\gamma$ is arbitrarily small, the $U.S.$ $\neg Spend$ and $China$ $Spend$s. When $\delta$ is arbitrarily large and $\gamma$ is arbitrarily large, neither player $Spend$s.

\section{Other Extras}
\subsection*{Theory} 
U.S. wants to protect American Big Tech markets that are potentially going to be lost in a zero-sum game with Chinese companies (if China successfully shares tech and builds infrastructure).

\begin{align*}
    H_1a:\; & \text{A pattern should be evident of U.S. money flowing into places where either}\\
    & \text{Big Tech markets are OR}\\
    H_1b:\; & \text{Where Chinese influence/money is increasing OR}\\
    H_2:\; & \text{Where Chinese influence is NOT increasing (U.S. defending markets)}
\end{align*}

\subsection*{Alternative Explanations}
1) The U.S. is not concerned necessarily about Big Tech markets, but instead increasing aid in an attempt to thwart China's goals of economic growth abroad.

\textbf{Implication}
\begin{align*}
    H_3:\; & \text{There should be a relatively random pattern of spending or at least}\\
    & \text{a pattern not associated with Big Tech markets}
\end{align*}

2) The U.S. is not necessarily concerned about market access for American Big Tech or China's goal of increasing economic growth through trade/economic diplomacy, but rather trying to counter Chinese money's influence on digital authoritarianism. The U.S. would then be trying to share it's own tech and build it's own internet infrastructure to have more control over the recipient's internet freedom levels.

\textbf{Implication}
\begin{align*}
    H_4:\; & \text{U.S. spending patters would be associated with increasing}\\
    & \text{internet freedom levels and potentially concentrated in places that are}\\
    & \text{not already receiving tech and infrastructure from China}\\
\end{align*}

9/28
If U.S. foreign aid policy is rational and related to internet freedom, then the U.S. must stand to gain from spending money on improving others' internet freedom. I argue that the U.S. benefits in two main ways from supporting internet freedom: protecting American technology's access to developing markets (possibly being encouraged by China to use Chinese technology instead) while ensuring that China faces resistance in continuing its economic growth (which is it's main goal of this project).

If this is true, then we should observe U.S. foreign aid spending patterns with one set of the following hypotheses:

\begin{align*}
    H_1a:\; & \text{U.S. aid should be associated with countries with high levels of internet freedom.}\\
    H_1b:\; & \text{U.S. aid should be associated with countries with low levels of internet freedom.}\\
    H_2a:\; & \text{U.S. aid should be associated with countries receiving high amounts of Chinese aid.}\\
    H_2b:\; & \text{U.S. aid should be associated with countries receiving low amounts of Chinese aid.}
\end{align*}

9/29
\subsection*{Theory}
Depending on how the literature review goes, we can mention our theory of why it is rational for the U.S. to engage in promoting internet freedom (aid effectually increases democracy, and democracy is good for business (American Big Tech)).
