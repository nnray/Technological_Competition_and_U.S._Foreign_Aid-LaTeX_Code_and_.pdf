Goal is to create a theory that rationalizes the U.S./G7's decision to increase aid to internet-related infrastructure (i.e., the U.S. says it is a value-driven effort to increase democratic democracy, but how is it rational for the U.S. to invest in that?) Counter-point: why is it surprising that donor countries, allegedly interested in economic development, would make internet infrastructure part of that development goal since the internet is good for the economy? That's certainly possible and could be one of my hypotheses... can I get information on the type of development most needed in recipient countries and use that to make a case that the internet should not be a priority?

What about USaid~Chinese aid + internet freedom levels? If USaid theoretically concerned with digital autocracy. See if there's a precedent for USaid being concerned with digital democracy. Use most recent data I can (e.g., OECD 2021) to see if during the Biden administration there has already been evidence of this despite majority of funds not being actualized yet (announcement in 2022).

Need to find extant theories in the literature on how U.S./other countries rationally distribute aid and see where my theory can contribute or not

Then, would need some way to grapple with the empirical problem or supporting any theoretical implications

We need to be more innocent about this project. We really don't know what is going on. There seem to be some alternative hypotheses: internet infrastructure good for recipient country and donors altruistic, recipients declare that is a need and donors responsive whether self-interested or not. Our hypothesis: donor countries interested in internet infrastructure to prevent China from overtaking West's digital access in these countries...

Debate in literature on if aid is self-interested, altruistic... If US was aiding digital democracy, is that a self-interested or altruistic act? 