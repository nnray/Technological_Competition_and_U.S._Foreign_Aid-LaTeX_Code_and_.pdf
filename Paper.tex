\documentclass{article}
\usepackage[utf8]{inputenc}
\title{\vspace{-2cm}Digital Authoritarianism and United States Foreign Aid\vspace{-0.5cm}}
\author{Nicholas Ray}
\date{\vspace{-0.5cm}September 2 2022\vspace{-1cm}}
\usepackage[margin=1in]{geometry}
\usepackage{mathtools,amssymb,amsthm}
\usepackage{setspace}
\doublespacing
\usepackage[backend=biber, style=authoryear, maxbibnames=99]{biblatex}
\renewbibmacro{in:}{}
\renewbibmacro*{volume+number+eid}{%
  \printfield{volume}
  \setunit*{\addnbthinspace}
  \printfield{number}
  \setunit{\addcomma\space}
  \printfield{eid}}
\DeclareFieldFormat[article]{number}{\mkbibparens{#1}}
\AtEveryBibitem{
  \clearfield{issn}
  \clearfield{month}
  \clearfield{urlyear}
  \clearlist{language}
  \clearfield{note}
  \ifentrytype{online}{}{
    \clearfield{url}
  }
}
\addbibresource{ForeignAidBib.bib}
\begin{document}
\maketitle
\section*{Introduction}
U.S.'s history of concern over global internet freedom. (Brookings, Rand, Congressional Bill, Declaration, Partnership for Global Infrastructure and Investment).
Given the U.S.'s history of concern over global internet freedom, how has U.S. foreign policy changed in response to China's involvement in developing countries? 
While there has been a wealth of academic attention on the potential effects of China's BRI (e.g., good journal articles), few have focused on how it might have shifted U.S. foreign policy. It is reasonable to expect that China's program would have \textit{some} effect on U.S. policy since 1) China's program is drawing attention from U.S. (Partnership, Declaration; and it could potentially be negatively affecting internet freedom) and 2) research has shown that U.S. aid is effective in promoting democracy despite Chinese aid prevalence (Blair, Marty, Roessler 2021).

\section*{Literature Review}
Most foreign aid allocated to self-interest and not recipient need (\cite{hoeffler_need_2011})

Aid ineffectiveness in achieving economic growth or promoting democratic institutions (Chong, Gradstein, Calderon)

Aid from U.S. successful in promoting democracy (Blair, Marty, Roessler 2021)

Aid from U.S. tends to be more effective in countries that receive no substantial support from China (so we should see funding directed towards countries lacking Chinese aid already)(Dreher et al 2021 AER)

Self-interest (Alesina and Dollar)

Chinese aid associated with increase in authoritarian values; World bank aid associated with democracy (Gehring, Kaplan, and Wong)

aid increases democracy (Dietrich, Wright)

Commercial self-interest present but possibly overblown (Dreher, Nunnenkamp, Thiele)

Bilateral aid more likely to be conditioned on things and less impartial, implying that U.S. should use it more if it is really concerned with internet freedom changes (Christensen, Homer, Nielson)

Trade benefits drive aid (Younas)

Political economic factors can shape preferences for foreign aid in U.S. (Milner, Tingley)

\section*{Methodology}
Dependent variable: US Aid
Main explanatory variables: Chinese Aid, Internet Freedom
Controls (according to Cinelli et al.): GDP per capita (recipient need), Internet Usership (Internet development indicators), Political Freedom (possible confounder)

Chinese data set (Custer et al. 2021; Dreher et al. 2022 (Banking on Beijing))

If concerned about simultaneity between US aid and Internet Freedom, maybe use instrument related to Internet Freedom but not to US aid. 
\section*{Complications}
 

\section*{Conclusion}
(Speculate about why US might be rationally interested in increasing internet freedom) Given the literature on the self-interestedness of donors, if U.S. aid is changing in response to Chinese aid, what might be a rational motivation? Basically: aid effectually increases democracy, democracy is good for business (American Big Tech).

\pagebreak
\printbibliography
\end{document}