\documentclass{article}
\usepackage[utf8]{inputenc}
\title{\vspace{-2.75cm}Digital Authoritarianism and United States Foreign Aid\vspace{-0.5cm}}
\author{Nicholas Ray}
\date{\vspace{-0.30cm}October 2 2022\vspace{-1cm}}
\usepackage[margin=1in]{geometry}
\usepackage{mathtools,amssymb,amsthm}
\usepackage{setspace}
\doublespacing
\usepackage[backend=biber, style=authoryear, maxbibnames=99,uniquelist=false]{biblatex}
\renewbibmacro{in:}{}
\renewbibmacro*{volume+number+eid}{%
  \printfield{volume}
  \setunit*{\addnbthinspace}
  \printfield{number}
  \setunit{\addcomma\space}
  \printfield{eid}}
\DeclareFieldFormat[article]{number}{\mkbibparens{#1}}
\AtEveryBibitem{
  \clearfield{issn}
  \clearfield{month}
  \clearfield{urlyear}
  \clearlist{language}
  \clearfield{note}
  \clearfield{month}
  \clearfield{day}
  \ifentrytype{online}{}{
    \clearfield{url}
  }
}
\addbibresource{ForeignAidBib.bib}
\begin{document}
\maketitle
\section*{Introduction}
The United States (U.S.) has been openly concerned about the internet since at least 1997, when President Bill Clinton exhorted the world to adopt a laissez-faire, commercially friendly approach to the internet (\cite{government1997}). By the 2000's, these preferences for a ``free'' internet were officially crystallized into foreign policy initiatives as so-called ``internet freedom'' programs cropped up in different facets of the U.S. government, such as the Department of State (e.g., \cite{henry2014}; \cite{government2021a}) and the U.S. Agency for Global Media (\cite{government2022}).

U.S. attention to internet freedom has continued to grow (e.g., \cite{government2010}), crescendoing most recently in 2022. All in one year, bipartisan legislation was introduced in the Senate to increase funding to internet freedom programs (\cite{government2022a}), a declaration was signed by the U.S. and over 60 other countries to combat global ``digital authoritarianism'' (\cite{government2022b}), and an unprecedented, U.S.-led G7 spending plan was unveiled to improve physical and digital infrastructure abroad (\cite{government2022c}). 

Along the way, the People's Republic of China (China) and the Russian Federation (Russia) have been named the antagonists to global internet freedom and the explicit motivations behind the U.S.'s focus on the subject (see \cite{government2010}, \cite{government2022b}). During this intensifying period of U.S. apprehension of apparent digital authoritarianism, China launched it's Belt and Road Initiative (BRI) in 2013 and the accompanying Digital Silk Road Initiative (DSR) in 2015. Collectively, these projects have catalyzed over one trillion U.S. dollars in developing-country investment with the intent to improve China's perception and influence in the world while fostering it's domestic economic growth (\cite{dreher2022}).

Ample academic work has recently attempted to outline the effects of China's aid program (e.g., \cite{blair2021}; \cite{eichenauer2021}; \cite{dreher2018}). Yet, little research has explored the potential effects of China's spending on U.S. foreign policy. In this paper, I seek to understand if U.S. foreign aid, an important aspect of foreign policy, can be understood as a response to China's aid ambitions. 

Literature shows that foreign aid is typically a self-interested tool (e.g., \cite{hoeffler2011}) and that U.S. aid can be effective in stimulating liberal democratic values in a contemporary environment saturated with Chinese aid (\cite{blair2022}). Given these findings, it seems likely that a view of U.S. foreign aid over time would reflect the historical policy importance of internet freedom and the U.S.'s desire to oppose both China's growing influence and their authoritarian model of the internet. Specifically, I hypothesize that U.S. foreign aid to Africa is related to a recipient's internet freedom level and their influx of Chinese aid.

Should this relationship be borne out in the data, future work would have to precisely explain how the U.S. benefits from higher global levels of internet freedom if foreign aid is indeed a self-interested tool. If no relationship is apparent, however, it must similarly be explained why the U.S. is failing to reconcile their foreign policy position on internet freedom with their intention to compete with China's expanding aid project since there is evidence that foreign aid can be effective in promoting liberal democratic values.

Understanding the U.S.'s actualization of internet preferences has important implications not only for the foreign aid literature, but also for international relations and comparative scholars more broadly. Firstly, any result that indicated the importance of internet considerations in inter-state behavior would be the first of its kind to my knowledge, and would signify a potentially novel dimension along which states interact. Secondly, this paper would help inform the dynamics of the U.S. and China's ongoing contest for economic and political influence by highlighting another issue area by which they compete. Lastly, this work would raise questions about the generalizability of state concern over the internet, which could encompass studies on conflict and cyberspace, research on democratization and information asymmetries, and political economic work on globalization.

I proceed by highlighting the relevant literature on foreign aid before moving on to the operationalization and testing of the hypothesis. In the concluding remarks, I offer tentative theories for various anticipated results.  

\section*{Literature Review}
\subsection*{Altruistic versus Self-Interested Aid}
Scholars have long debated if donors of foreign aid are motivated by altruism or self-interest. The received wisdom on foreign aid states that donors often give aid according to their strategic concerns (e.g., \cite{mckinlay1977}; \cite{alesina2000}), and economic trade has more recently been shown to be a predictor of donor aid. Specifically, donors seem to distribute aid amongst recipients that import goods that donors have a comparative advantage in producing (\cite{younas2008}).

However, some studies have implied altruism is also an important factor in determining aid flows, especially for private aid and non-governmental organizations (\cite{buthe2012}). Other scholars have analyzed the potential for aid to reduce poverty, concluding that poverty reduction is occurring but in an inefficient manner (\cite{collier2002}). The consensus, though, still seems to be that recipient humanitarian need is largely dominated by donor self-interest (e.g., \cite{hoeffler2011}).

The balance between strategic interests and altruism has additionally been conceived of along the lines of donor maturity, with more established donors potentially distributing aid in more or less altruistic ways than ``new'' donors (\cite{dreher2011}). Generally, though, it is found that traditional aid providers care less for recipient need than previously thought while relatively young donors are less self-interested than expected by previous literature.

\subsection*{Foreign Aid: An Effective Tool?}
Regardless of the exact source of motivation behind foreign aid, there has been a significant amount of disagreement over whether foreign aid is effective in achieving any objectives. In some cases, for example, it appears that the lack of discernible and consistent motivations is the very reason why aid can be found to be ineffective (\cite{collier2002}). Aid has also been found to be ineffective in altering levels of inequality or poverty in recipient countries, even if those countries have relatively strong institutions (\cite{chong2009a}).

There is, however, modest evidence that economic aid effectively contributes to democratic consolidation in Africa (\cite{dietrich2015a}). Effectiveness can vary by certain characteristics of aid as well, with bilateral aid being more likely to be conditioned on a recipient country's level of corruption. Multilateral aid, on the other hand, is generally more impartial and unconditional, resulting in diminished effectiveness when compared to bilateral aid (\cite{christensen2011}).

Interestingly, evidence for the effectiveness of aid appears to have strengthened with China's monumental entrance into the aid arena. For example, when comparing the effects of Chinese and U.S. aid in Africa, U.S. aid is found to be relatively successful in promoting liberal democratic values amongst Afrobarometer respondents (\cite{blair2022}). Chinese aid, by contrast, largely has null or even similarly positive effects on democratic values. In a separate study relating World Bank and Chinese aid, foreign aid from the World Bank is associated with the same liberal democratic values and even improved country stability. A feeling of acceptance towards autocratic leadership is found to be connected to Chinese aid (\cite{gehring2022}).

Yet there is a substantial amount of nuance to aid effectiveness. Aid from China is definitively linked with positive economic growth in recipient countries, for example, while U.S. aid is most effective in countries not currently receiving large amounts of Chinese aid (\cite{dreher2021}).

Thus, it can be gathered from previous literature that donor behavior, including that of the U.S., should be expected to be a product of strategic considerations. Furthermore, while not always effective in securing these considerations, it is plausible that the current atmosphere of aid-giving in Africa represents a scenario in which aid tied to liberal democratic ideals can be effectual. 

\section*{Empirics}
\subsection*{Data and Variables}
In approaching an analysis of the relationship between foreign aid and internet freedom, it perhaps makes most sense to firstly define the main explanatory variable of internet freedom. The Digital Society Project (DSP) aims to identify ``how people use social media as a political tool and to explore how political institutions and social media use interact,'' providing data from expert surveys that cover a range of topics such as censorship, disinformation, and monitoring (\cite{mechkova2022}). The DSP employs methodology designed by the Varieties of Democracy project (V-Dem, \cite{coppedge2022}) to try and estimate differences in expert opinion and knowledge.

Data from the DSP is potentially superior for measuring the concept of internet freedom compared to the primary alternative, Freedom on the Net (FOTN, \cite{house2022}). FOTN features frequent missing data for the continent of Africa and does not take into account possible variations in expert opinion and knowledge like the DSP does.

The other main explanatory variable is Chinese foreign aid, stemming from the College of William \& Mary's AidData research lab (\cite{custer2021}). This is the one of the only datasets on Chinese aid and is certainly the largest and most specific. Aid is quantified as official financial and in-kind commitments and is measured in constant 2017 USD.

U.S. foreign aid is the dependent variable, measured as fiscal-year disbursements (\cite{government2022e}). It is measured in constant 2022 USD.

The starting year for the holistic dataset is 2000, defined by the lower bound of data availability for internet freedom, and 2017 is the ending year per data availability for Chinese foreign aid. There is some evidence of left-truncation or missing data in the information on internet freedom, but otherwise all African countries are accounted for.

\subsection*{Methodology}
To estimate the association between U.S. foreign aid and the main explanatory variables, I plan to use ordinary least squares (OLS) with country fixed-effects. Fixed-effects at the country level help account for any underlying similarity between countries that could induce spuriousness.

In addition to including fixed-effects, there are multiple controls that seem appropriate to incorporate. According to the most recent advice for determining good controls (\cite{cinelli2022}), I include gross domestic product (GDP) per capita, an indicator for internet development, and political freedom as controls. Each of these phenomena are possible confounders, being theoretically related to at least one of the types of aid and internet freedom. 

GDP per capita is assessed in constant 2021 USD and as is supplied by the World Bank (\cite{bank2022}). The indicator for internet development, internet usage by country, is calculated as a percentage and is provided by the International Telecommunication Union (\cite{itu2022}). Lastly, the chosen measure for political freedom comes from V-Dem (\cite{coppedge2022a}).

\section*{Limitations}
There are at least a couple important limitations to be clear about. Firstly, there is undoubtedly some measurement error in the quantification of internet freedom brought about due to possible differences in the idea of internet freedom employed by the U.S. and DSP. In other words, even if the U.S.'s conception of internet freedom was constant over time, the DSP definition is likely to be somewhat distinct.

Secondly, there is potential simultaneity between U.S. foreign aid and internet freedom. More advanced econometric modeling techniques, such as difference-in-differences (DiD) or instrumental variables regression (IV) may be imperative here. 

\section*{Conclusion}
It is possible that there is no meaningful association between U.S. foreign aid and internet freedom. The lack of association would in itself present a sort of puzzle, though, since the U.S. clearly has a foreign policy interest in maintaining internet freedom and a seemingly effective tool of foreign aid to pursue that interest. It may take time, however, for the ever-increasing advances in U.S. financial commitment to internet freedom to be reflected in spending data.  

If there is indeed a relationship, I speculate that the U.S. is interested in keeping the internet free because it is good for business. There is evidence that China is sharing it's technology, like censorship and monitoring tools, that could allow more authoritarian-leaning developing countries to prevent the use of U.S. technology and to strictly allow for Chinese substitutes instead. These developing countries would then harness the economic benefits of the internet, like job growth, while constraining potentially negative political externalities such as the free assembly of anti-incumbent online groups. China would stand to gain domestic economic growth as it's technology firms expand beyond it's borders, and everybody wins- except the U.S.

\nocite{mechkova2022a}
\nocite{pemstein2022}
\nocite{coppedge2022}
\nocite{coppedge2022a}
\nocite{coppedge2022b}
\nocite{coppedge2022c}
\nocite{coppedge2022d}
\pagebreak
\printbibliography
\end{document}