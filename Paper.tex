\documentclass{article}
\usepackage[utf8]{inputenc}
\title{Digital Authoritarianism and Western Foreign Aid}
\author{Nicholas Ray}
\date{September 2022}
\usepackage[margin=1in]{geometry}
\usepackage{mathtools,amssymb,amsthm}
\usepackage{setspace}
\doublespacing
\usepackage[backend=biber, style=authoryear, maxbibnames=99]{biblatex}
\renewbibmacro{in:}{}
\renewbibmacro*{volume+number+eid}{%
  \printfield{volume}
  \setunit*{\addnbthinspace}
  \printfield{number}
  \setunit{\addcomma\space}
  \printfield{eid}}
\DeclareFieldFormat[article]{number}{\mkbibparens{#1}}
\AtEveryBibitem{
  \clearfield{issn}
  \clearfield{month}
  \clearfield{urlyear}
  \clearlist{language}
  \clearfield{note}
  \ifentrytype{online}{}{
    \clearfield{url}
  }
}
\addbibresource{References.bib}
\begin{document}
\maketitle
\section*{Introduction}
(What needs to be explained? What is the puzzling observation? What is the research question?)
Has China's aid project changed how the United States conducts it's aid? I argue that we should expect the Chinese project to have an effect because the United States has made internet freedom part of it's foreign aid policy, which is perhaps being affected by Chinese aid itself. Been a lot of interest recently on the effects of China's BRI (e.g., cite some highly published journals).

Evidence that internet freedom has been a part of U.S. foreign policy for the past decade, but no observational evidence that U.S. foreign aid (an important part of U.S. foreign policy) is related to the internet freedom of recipient countries. Further, it is not ex ante obvious why the U.S. should rationally be spending money on a relatively abstract concept of internet freedom (literature on how concern for democracy drives U.S.? There's a lot of literature on if the US's aid program is self-interested or altruistic, which can leverage here (i.e., given that U.S. seems to be self-interested, why?))

I argue that to the extent that U.S. aid is related to internet freedom, it is rational for the U.S. to expend resources on increasing internet freedom because 1) it secures them access for the future to developing country's markets that could give profitable personal data to U.S. Big Tech (there is a possibility that internet is a zero-sum game between China, U.S., where U.S. wants to preempt China's takeover of technology markets, i.e. technological arms race, access war) and 2) the U.S. has an interest in thwarting China's efforts at increasing domestic economic growth through their technology companies.

\section*{Literature Review}
\subsection*{Determinants of Aid}
\textbf{Self-Interest versus Altruism}

Hoeffler, Outram: ``Our results indicate that all bilateral donors allocate aid according to their self-interest and recipient need''

Younas: Trade seems to account for donor behavior, along with priorities in human rights and miseries rather than poverty.

Wiseman, Young: Donors base behavior off of ``warm-glow''

National interests versus recipient need (general debate about self-interest or in the interest of recipients).


\section*{Theory}
It is rational for the U.S. to (condition?) spend foreign aid according to internet freedom levels of recipient because...

\section*{Methodology}


\section*{Complications}
 

\section*{Conclusion}


\pagebreak
\printbibliography
\end{document}