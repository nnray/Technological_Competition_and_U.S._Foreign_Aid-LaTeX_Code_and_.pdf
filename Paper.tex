\documentclass{article}
\usepackage[utf8]{inputenc}
\title{Digital Authoritarianism and Western Foreign Aid}
\author{Nicholas Ray}
\date{September 2022}
\usepackage[margin=1in]{geometry}
\usepackage{mathtools,amssymb,amsthm}
\usepackage{setspace}
\doublespacing
\usepackage[backend=biber, style=authoryear, maxbibnames=99]{biblatex}
\renewbibmacro{in:}{}
\renewbibmacro*{volume+number+eid}{%
  \printfield{volume}
  \setunit*{\addnbthinspace}
  \printfield{number}
  \setunit{\addcomma\space}
  \printfield{eid}}
\DeclareFieldFormat[article]{number}{\mkbibparens{#1}}
\AtEveryBibitem{
  \clearfield{issn}
  \clearfield{month}
  \clearfield{urlyear}
  \clearlist{language}
  \clearfield{note}
  \ifentrytype{online}{}{
    \clearfield{url}
  }
}
\addbibresource{References.bib}
\begin{document}
\maketitle
\section*{Introduction}
(What needs to be explained? What is the puzzling observation? What is the research question?)

Evidence that internet freedom has been a part of U.S. foreign policy for the past decade, but no empirical evidence that U.S. foreign aid 
(an important part of U.S. foreign policy) is driven by internet freedom concerns. If it is, why is it rational for the U.S. to be concerned about internet freedom? (Further, maybe, does U.S. aid have a positive effect on internet freedom?) I argue that to the extent that U.S. aid is related to internet freedom, it is rational for the U.S. to expend resources on increasing internet freedom because 1) it secures them access for the future to developing country's markets that could give profitable personal data to U.S. Big Tech and 2) the U.S. has an interest in thwarting China's efforts at increasing domestic economic growth through their technology companies.

\section*{Literature Review}
\subsection*{Determinants of Aid}
\textbf{Self-Interest versus Altruism}

Hoeffler, Outram: ``Our results indicate that all bilateral donors allocate aid according to their self-interest and recipient need''

Younas: Trade seems to account for donor behavior, along with priorities in human rights and miseries rather than poverty.

Wiseman, Young: Donors base behavior off of ``warm-glow''

National interests versus recipient need (general debate about self-interest or in the interest of recipients).


\section*{Theory}


\section*{Methodology}


\section*{Complications}
 

\section*{Conclusion}


\pagebreak
\printbibliography
\end{document}