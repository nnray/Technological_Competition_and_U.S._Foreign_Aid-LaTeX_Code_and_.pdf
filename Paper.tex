\documentclass{article}
\usepackage[utf8]{inputenc}
\title{\vspace{-2cm}Digital Authoritarianism and United States Foreign Aid\vspace{-0.5cm}}
\author{Nicholas Ray}
\date{\vspace{-0.5cm}September 2 2022\vspace{-1cm}}
\usepackage[margin=1in]{geometry}
\usepackage{mathtools,amssymb,amsthm}
\usepackage{setspace}
\doublespacing
\usepackage[backend=biber, style=authoryear, maxbibnames=99,uniquelist=false]{biblatex}
\renewbibmacro{in:}{}
\renewbibmacro*{volume+number+eid}{%
  \printfield{volume}
  \setunit*{\addnbthinspace}
  \printfield{number}
  \setunit{\addcomma\space}
  \printfield{eid}}
\DeclareFieldFormat[article]{number}{\mkbibparens{#1}}
\AtEveryBibitem{
  \clearfield{issn}
  \clearfield{month}
  \clearfield{urlyear}
  \clearlist{language}
  \clearfield{note}
  \clearfield{month}
  \clearfield{day}
  \ifentrytype{online}{}{
    \clearfield{url}
  }
}
\addbibresource{ForeignAidBib.bib}
\begin{document}
\maketitle
\section*{Introduction}
The United States (U.S.) has been openly concerned about the internet since at least 1997, when President Bill Clinton exhorted the world to adopt a laissez-faire, commercially friendly approach to the internet (\cite{government1997}). By the 2000's, these preferences for a ``free'' internet were officially crystallized into foreign policy initiatives as so-called internet freedom programs cropped up in different facets of the U.S. government, such as the Department of State (e.g., \cite{henry2014}; \cite{government2021a}) and the U.S. Agency for Global Media (\cite{government2022}).

U.S. attention to internet freedom has continued to grow (e.g., \cite{government2010}), crescendoing most recently in 2022. All in one year, bipartisan legislation was introduced in the Senate to increase funding to internet freedom programs (\cite{government2022a}), a declaration was signed by the U.S. and over 60 other countries to combat global ``digital authoritarianism'' (\cite{government2022b}), and an unprecedented, U.S.-led G7 spending plan was unveiled to improve physical and digital infrastructure abroad (\cite{government2022c}). 

Along the way, the People's Republic of China (China) and the Russian Federation (Russia) have been named the antagonists to global internet freedom and the explicit motivations behind the U.S.'s focus on the subject (see \cite{government2010}, \cite{government2022b}). During this intensifying period of U.S. apprehension of apparent digital authoritarianism, China launched it's Belt and Road Initiative (BRI) in 2013 and the accompanying Digital Silk Road Initiative (DSR) in 2015. Collectively, these projects have catalyzed over one trillion U.S. dollars in developing-country investment with the intent to improve China's perception and influence in the world while fostering it's domestic economic growth (\cite{dreher2022}).

Ample academic work has recently attempted to outline the effects of China's aid program (e.g., \cite{blair2021}; \cite{eichenauer2021}; \cite{dreher2018}). Yet, little research has explored the potential effects of China's spending on U.S. foreign policy. In this paper, I seek to understand if U.S. foreign aid, an important aspect of foreign policy, can be understood as a response to China's aid ambitions. 

Literature shows that foreign aid is typically a self-interested tool (e.g., \cite{hoeffler2011}) and that U.S. aid can be effective in stimulating liberal democratic values in a contemporary environment saturated with Chinese aid (\cite{blair2022}). Given these findings, it seems likely that a view of U.S. foreign aid over time would reflect the historical policy importance of internet freedom and the U.S.'s desire to oppose China's growing influence and authoritarian model of the internet. Specifically, I hypothesize that U.S. foreign aid to Africa is related to a recipient's internet freedom level and their influx of Chinese aid.

Should this relationship be borne out in the data, future work would have to precisely explain how the U.S. benefits from higher global levels of internet freedom if foreign aid is indeed a self-interested tool. If no relationship is apparent, however, it must similarly be explained why the U.S. is failing to reconcile their foreign policy position on internet freedom with their intention to compete with China's expanding aid project since there is evidence that foreign aid can be effective in promoting liberal democratic values.

I proceed by highlighting the relevant literature on foreign aid before moving on to the operationalization and testing of the hypothesis. In the concluding remarks, I offer tentative theories for various anticipated results.  

\section*{Literature Review} %improve literature review: both the number and quality of discussion on citations; it is imperative that I can claim that U.S. interest in internet freedom makes sense in terms 1) self-interest and 2) effectiveness in the current aid climate. Also, maybe contain a few citations about how the U.S. as a specific case creates foreign aid preferences (e.g., Milner and folks)
\subsection*{Determinants of Foreign Aid: Self-Interest or Altruism?}
Scholars have long debated if donors of foreign aid are motivated by altruism or self-interest. The received wisdom on foreign aid states that donors often give aid according to their strategic interests (e.g., \cite{mckinlay1977}), and economic trade has been shown to be a predictor of donor aid. Specifically, donors seem to distribute aid amongst recipients that import goods that donors have a comparative advantage in producing (\cite{younas2008}).

However, some studies have implied that altruism can be influential as well. France and Japan, for example, have displayed patterns of spending associated with past colonial ties and UN votes, respectively (\cite{alesina2000}). Recipient humanitarian need is also thought to be important in determining aid flows and that, despite inefficiencies in allocation, aid still results in substantial poverty reduction (\cite{collier2002}). However, more recent work on recipient need has seemed to conclude that it is largely dominated by donor self-interest, although donor behavior is still poorly understood (\cite{hoeffler2011}.

There is also seeming variation amongst donor countries depending on if they are a part of the Development Assistance Committee (DAC). More traditional donors that operate with DAC are shown to care more for recipient need while nonDAC donors allocate aid more so with regard to self-interest (\cite{dreher2011}). Generally, though, DAC aid providers care less for recipient need than previously thought while nonDAC members are less self-interested than expected by previous literature.

\subsection*{Foreign Aid: An Effective Tool?}
Regardless of the motivations behind foreign aid, there has been a significant amount of disagreement over whether foreign aid is effective in achieving its objectives. Aid, for example, has been found to be ineffective in altering levels of inequality or poverty in recipient countries, even if those countries have relatively strong institutions (\cite{chong2009a}). The lack of intentionality behind aid allocation is thought to contribute to its ineffectiveness, with donors not necessarily interested in maximizing aid effectiveness on poverty reduction or humanitarian need (\cite{alesina2000}, \cite{collier2002}).

There is, however, modest evidence that economic aid contributes to democratic consolidation in Africa (\cite{dietrich2015a}).

Interestingly, evidence for the effectiveness of aid appears to have increased with China's monumental entrance into aid provision. For example, when comparing the effects of Chinese and U.S. aid in Africa, U.S. aid is relatively successful in promoting liberal democratic values amongst Afrobarometer respondents (\cite{blair2022}). Chinese aid, by contrast, largely has null or even positive effects on democratic values as well.

Further research on aid's impact on democratic preferences in Africa after the rise of Chinese aid finds that aid from the World Bank similarly increased liberal democratic attitudes (\cite{gehring2022}). World Bank aid is even associated with improved stability, whereas Chinese aid is shown to have relatively smaller effects on stability and a positive relationship to autocratic acceptance.

There is a substantial amount of nuance to aid effectiveness, though. Aid from China is associated with positive economic growth in recipient countries while U.S. aid is relatively more effective in countries not currently receiving Chinese aid (\cite{dreher2021}).

Other work details how bilateral aid is more likely to be conditioned on levels of corruption and thus more effective than multilateral aid (\cite{christensen2011}).

\section*{Methodology}
Following the logic and stated hypothesis in the introduction, the dependent variable is U.S. foreign aid to countries in Africa while the main explanatory variables are recipient country internet freedom levels and Chinese foreign aid to Africa. The time-span of the data range from 2000-2017, with the data described as follows.

U.S. foreign aid is measured using disbursement data from the U.S. government (\cite{government2022e}), which reports foreign aid on a fiscal-year basis. The measurement for Chinese foreign aid stems from the College of William \& Mary (\cite{custer2021}) which quantifies official financial and in-kind commitments made by China. Internet freedom data is provided by the Digital Society Project (DSP, \cite{mechkova2022}), being collected through expert surveys and estimated with methodology designed by the Varieties of Democracy project (V-Dem, \cite{coppedge2022}). The DSP dataset is potentially superior for measuring the concept of internet freedom compared to the alternative, Freedom of the Net (FOTN, \cite{house2022}). FOTN features frequent missing data for the continent of Africa and does not take into account possible variations in expert opinion and knowledge like the DSP does.

According to the most recent advice for determining good controls (\cite{cinelli2022}), I include GDP per capita, an indicator for internet development and political freedom as controls. GDP per capita is assessed in terms of 2021 U.S. dollars as is supplied by the World Bank (\cite{bank2022}). The indicator for internet development, internet usage by country, is calculated as a percentage and is reported by the International Telecommunication Union (ITU, \cite{itu2022}). Lastly, the chosen measure for political freedom comes from V-Dem (\cite{coppedge2022a}).

OLS with country fixed effects

IV?

Case study comparing U.S. aid changes to countries that had similar internet freedom levels but experienced different influxes of Chinese aid.

\section*{Limitations}
Definition of internet freedom (difference between DSP, U.S.)

Endogeneity.

\section*{Conclusion}
Restate important parts of the introduction.

Speculate on theoretical reasons why U.S. aid might or might not be related to internet freedom and Chinese foreign aid. If positively related, say that internet freedom is good for capitalist business. If null, perhaps data doesn't capture recent trends as focus has increased. Also, perhaps government decision making slow to react (Congress) or unwilling for domestic reasons (Milner stuff, for example).

\nocite{mechkova2022a}
\nocite{pemstein2022}
\nocite{coppedge2022}
\nocite{coppedge2022a}
\nocite{coppedge2022b}
\nocite{coppedge2022c}
\nocite{coppedge2022d}
\pagebreak
\printbibliography
\end{document}